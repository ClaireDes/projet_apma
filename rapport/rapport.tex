\documentclass[a4paper,11pt]{article}
% \documentstyle[11pt]{article}
\usepackage[utf8]{inputenc} % permet de mettre les accents (UTF8)
\usepackage{graphicx}
% \pagestyle{headings}

\oddsidemargin 1 cm
\evensidemargin 1 cm
\textwidth 15 cm
\topmargin 0 cm
\textheight 22.5 cm

\begin {document}

\title {Jeu à deux joueurs à somme nulle et à information complète \\
Application au Puissance4}

\author{Claire DESHAYES \\ François COMBE}
\date{\today}
\maketitle

\section*{Introduction}

Nous avons choisi de réaliser le projet de jeu à somme nulle
et à information complète appliqué puissance 4. Nous nous sommes
donc dans un premier tempsconcentré sur les packages {\tt Partie}
et {\tt Puissance4}. \\ \\
Une fois le jeu fonctionnant pour deux joueurs humains nous avons
entrepris d'implémenter une IA capable de jouer (et surtout de bien
jouer) au Puissance4.

\section{Implémantation du jeu Puissance4}

\subsection{Le package {\tt Partie}}

Ce {\tt package} ne contient qu'une procédure {\tt Joue\_Partie} qui en
fonction de l'état actuel du jeu va demander au joueur à qui c'est le
tour de jouer, afficher le jeu, regarder si le joueur à gagné ou si la partie
est nulle, et fonction annoncer le vainqueur ou s'appeler récursivement pour
faire jouer l'adversaire. \\
Pour cela, la procédure {\tt Joue\_Partie} utilise les fonctions définies dans
le {\tt package Puissance4} et s'occupe de gérer le déroulement de la partie.\\
\\
Nous n'avons pas eu à faire énormément de choix pour cette partie. En
effet il s'agit d'un enchainement d'appels aux fonctions adaptées.
Le seul point que nous aurions pu éventuellement amélioré est la distinction
entre les deux joueurs. Il était peut-être possible d'améliorer notre code
de ce point de vue afin d'éviter les redondances mais nous avons préféré
donner la priorité à d'autres points.














\end {document}
