\documentclass[a4paper,11pt]{article}
% \documentstyle[11pt]{article}
\usepackage[utf8]{inputenc} % permet de mettre les accents (UTF8)
\usepackage{graphicx}
% \pagestyle{headings}

\oddsidemargin 1 cm
\evensidemargin 1 cm
\textwidth 15 cm
\topmargin 0 cm
\textheight 22.5 cm

\begin {document}

\title {Jeu à deux joueurs à somme nulle et à information complète \\
Application au Puissance4}

\author{Claire DESHAYES \\ François COMBE}
\date{\today}
\maketitle

\section*{Introduction}

Nous avons choisi de réaliser le projet de jeu à somme nulle
et à information complète appliqué puissance 4. Nous nous sommes
donc dans un premier tempsconcentré sur les packages {\tt Partie}
et {\tt Puissance4}. \\ \\
Une fois le jeu fonctionnant pour deux joueurs humains nous avons
entrepris d'implémenter une IA capable de jouer (et surtout de bien
jouer) au Puissance4.

\section{Implémentation du jeu Puissance4}

\subsection{Le package {\tt Partie}}

Ce {\tt package} ne contient qu'une procédure {\tt Joue\_Partie} qui en
fonction de l'état actuel du jeu va demander au joueur à qui c'est le
tour de jouer, afficher le jeu, regarder si le joueur à gagné ou si la partie
est nulle, et fonction annoncer le vainqueur ou s'appeler récursivement pour
faire jouer l'adversaire. \\
Pour cela, la procédure {\tt Joue\_Partie} utilise les fonctions définies dans
le {\tt package Puissance4} et s'occupe de gérer le déroulement de la partie.\\
\\
Nous n'avons pas eu à faire énormément de choix pour cette partie. En
effet il s'agit d'un enchainement d'appels aux fonctions adaptées.
Le seul point que nous aurions pu éventuellement amélioré est la distinction
entre les deux joueurs. Il était peut-être possible d'améliorer notre code
de ce point de vue afin d'éviter les redondances mais nous avons préféré
donner la priorité à d'autres points.

\subsection{Le package {\tt Participant}}

Ce {\tt package} est très simple et n'a pas nécessité que l'on y passe beaucoup
de temps. La fonction {\tt Adversaire} retourne le joueur adverse du joueur
passé en paramètre. Et le type {\tt Joueur} est défini comme un entier
tel un {\tt enum} en language C qui peut prendre trois valeurs : Joueur1
, Joueur2 et Vide (nécessaire pour le package {\tt Puissance4}).
Cela apporte de la lisibilité au code.

\subsection{Le package {\tt Puissance4}}

C'est le {\tt package} qui a demandé le plus travail pour cette partie.
Il contient les fonctions et les procédures nécessaires à {\tt Joue\_Partie}.
Une partie de Puissance4 est définie par le nombre de lignes et de colonnes
du jeu ainsi que le nombre de pions à aligner pour gagner. \\
Nous avons défini le type {\tt Etat} comme un tableau à deux dimensions de
{\tt Joueur} représentant le jeu ({\it Pour un Etat E, E(i,j) vaut Joueur1 si
le joueur 1 a joué sur cette case}). \\
Le type {\tt Coup} est défini comme une structure contenant un entier {\tt Col}
représentant la colonne dans laquelle on joue et le {\tt Joueur J} qui indique
quel joueur à jouer ce coup.\\
\\
La procédure {\tt Initialiser} met toute les cases de l'état passé en paramètre
à {\tt Vide}. \\
La fonction {\tt Jouer} applique le coup à l'état, tout deux
passés en paramètres, et retourne l'état. Dans le cas où le coup n'est pas
possible (c'est à dire que la colonne est pleine) la fonction retourne l'état
précedent même si ce cas est normalement évité en amont. \\
La fonction {\tt Est\_Gagnant} retourne le booléen {\tt true} si le joueur passé
en paramètre à aligné le nombre de pions nécessaires. La fonction parcours dans
premier temps les lignes, puis les colonnes, et enfin les diagonales. Afin
de faire un parcours des différentes diagonales possibles "intelligent" nous avons
utilisé deux variables {\tt nb\_diags} pour le nombre de diagonales à sur
lesquelles on peut aligner le nombre de pions nécessaires et {\tt c\_diags}
qui peut s'apparenter un indice des diagonales et qui est utile pour
l'algorithme de parcours des cases. En conséquence, si le code est peut-être
difficile à lire, il est efficace est répond à toute les configurations de jeu
possibles.\\
La fonction {\tt Est\_Nul} retourne simplement le booléen {\tt true} si le jeu
est plein (il n'y a plus de case vide) et que les deux joueurs n'ont pas gagné
(ce que l'on vérifie avec des appels à la fonction {\tt Est\_Gagnant}). \\
Les procédures {\tt Afficher} et {\tt Affiche\_Coup} affichent respectivement
une représentation du jeu et le dernier coup joué d'une façon très claire. \\
Enfin les fonctions {\tt Demande\_Coup\_Joueur} créent un {\tt Coup}, demandent
au joueur dans quelle colonne il veut jouer et retournent le coup. C'est à ce niveau
que l'on gère le cas où on joue dans une colonne pleine. Dans ce cas on
redemande au même joueur de jouer à nouveau.\\
\\
C'est ce module qui nous a prit le plus de temps et a été le plus compliqué
à mettre en place. En effet nous avions dans un premier temps prit la décision
de représenter un {\tt Etat} par un tableau en une dimension. Nous avons eu
ainsi beaucoup de difficulté à trouver un algorithme parcourant les diagonales.
Suite à quoi nous avons modifié notre projet pour qu'il soit tel qui l'est
aujourd'hui.

\subsection{L'algorithme MinMax}

Dans cette partie nous avons le squelette du package Moteur_Jeu qui nous est donn�. Il s'agissait donc de le d�velopper puis de l'impl�menter, ce que nous avons fait gr�ce � un package {\tt main1joueur}. On y instancie donc le package Puissance 4 de la m�me fa�on que main2joueur. Ensuite on instancie le package Moteur_Jeu, pour �viter de r�-�crire des parties de code nous avons fait le choix d'ajouter au package puissance4 les fonctions et proc�dures n�cessaire pour l'instanciation du package Moteur_Jeu.
Enfin nous avons choisi d'instancier la partie en changeant simplement la fonction Demande_Coup_Joueur2 par Choix_Coup, pour ne pas demander au deuxi�me joueur de jouer mais laisser l'ordinateur choisir et jouer son coup.

Cependant, notre premi�re id�e �tant de cr�er un package moteur_jeu_puissance4qui impl�mente toutes les fonctions et proc�dures n�cessaires � l'instanciation d'un package moteur_jeu pour le puissance 4, notre changement nous a pos� des probl�mes de visibilit�s entre les packages.
Aussi nous avons �t� oblig� de changer la signature la fonction Eval pour quelle prenne en param�tre le joueur moteur car il n'�tait pas visible dans le package puissance4.

\subsection{Le package {\tt Moteur_jeu}}

Dans ce package nous devions mettre en place l'algorithme Min Max dans la fonction Choix_Coup.Pour cela nous parcourons les coups atteignable en un coup et nous �valons le gain de chaque coup gr�ce � la fonction Eval_Min_Max. On retourne le meilleur coup.

Dans la fonction Eval_Min_Max, nous avons choisi de parcourir en profondeur sur un entier P_Courant. Quand un �tat terminal est atteint on renvoie le gain maximum 1000 en cas de victoire, son oppos� en cas de d�faite et z�ro pour un match nul. Si l'�tat n'est pas terminal applique par r�currence la fonction Eval_Min_Max aux coups atteignables au prochain tour selon le principe Min/Max.





















\end {document}
